\documentclass[10pt,a4paper]{article}
\usepackage[letterpaper,top=2cm,bottom=2cm,left=3cm,right=3cm,marginparwidth=1.75cm]{geometry}
\usepackage{amsmath}
\usepackage{graphicx}
\usepackage{url}
\usepackage{float}
\geometry{margin=1.8cm}

\usepackage{listings}
\usepackage{xcolor}

\lstdefinestyle{matlabstyle}{
	language=Matlab,
	basicstyle=\ttfamily\small,
	numbers=left,
	numberstyle=\tiny,
	backgroundcolor=\color{gray!5},
	frame=single,
	keywordstyle=\color{blue},
	commentstyle=\color{green!50!black},
	stringstyle=\color{orange},
	breaklines=true
}

\usepackage{hyperref}
\hypersetup{
	colorlinks=true,
	linkcolor=blue,
	filecolor=magenta,      
	urlcolor=cyan,
	pdftitle={Overleaf Example},
	pdfpagemode=FullScreen,
}


\title{Simulation of Mechatronic Systems \\\large Exercise 5 protocol - Controller design for motor speed controller}
\IfFileExists{authors.tex}{
	% authors.tex
\author{
	Alessandro Rotondi - 27445613 \\
	Frederic Heil - 28742953 \\
	Nithin Badri - 17546426 \\
	Sudeesh Suragari - 14646468
} % this pulls in authors.tex
}{
	\author{please create an authors.tex file}
}

\begin{document}
	\maketitle
	\section*{Exercise 5a): Simulating the motor-behavior for an input-voltage of 10 Volts}
	In the previous exercise 4 we defined and built an open-loop model of a dc-motor. Now we want to simulate it with an input-voltage of 10 volts. For this, we can simply adjust the input voltage within our previously defined Simulink model, which we are also going to use for the follow-up tasks of this exercise. The model is shown below in figure \ref{fig:openLoopModelSimulink}:\newline\newline
	\noindent
	\textbf{Simulink Model:}
	\begin{figure}[H]
		\centering
		\includegraphics[width=0.8\linewidth]{openLoopSimulink.png}
		\caption{Open loop model of a dc-motor.}
		\label{fig:openLoopModelSimulink}
	\end{figure}
	\noindent
	In addition to that, the dc-motor's fixed physical propertied remain unchanged.
	\begin{itemize}
		\item $L=0.23 \cdot 10^{-3}H$
		\item $c_m = 23.4 \cdot 10^{-3} \frac{Nm}{A}$
		\item $R = 2.4 \Omega$
		\item $J = 0.23 \cdot 10^{-6} Nm^{2}$
		\item $D = 0.4191 \cdot 10^{-5} \frac{Ns}{m}$
	\end{itemize}
	\noindent
	Running this simulation with 10 volts yields the following step response equivalent to the previous exercise:
	\begin{figure}[H]
		\centering
		\includegraphics[width=0.8\linewidth]{step10vOpenLoop.pdf}
		\caption{Step response of the open loop model of a dc-motor at 10V.}
		\label{fig:step10vOpenLoop}
	\end{figure}
	\noindent
	As shown above in \ref{fig:step10vOpenLoop}, angular speed $\omega$ converges to a value just above $400 \frac{rad}{s}$ and the current $i$ to about $0.1A$. Time to convergence is about 0.005 seconds, which is similar to the result of exercise 4. Using our Simulink model, we can design a controller using the Linearize-Tool and Sisotool of Matlab in the next sub task.
	
	\section*{Exercise 5b): Simulating the motor-behavior for an input-voltage of 10 Volts}
	As visible in \ref{fig:openLoopModelSimulink}, the voltage input has been marked as \verb|Input Pertubation| and the output angular velocity $\omega$ has been marked as \verb|Output Measurement|. This equates to the first step in designing a speed controller in Matlab using the Linearize-Tool and Sisotool of Matlab.\newline
	With this step complete, we can open up the Linearize-Tool using Model-Linearizer in the Apps section. In this tool we generate a step response which yields \verb|linsys1|, which we can export to our Matlab workspace. The step response can be seen below in figure \ref{fig:Linearize}:
	
	\begin{figure}[H]
		\centering
		\includegraphics[width=1.0\linewidth]{linearizer.pdf}
		\caption{Step response created within the Linearize-Tool.}
		\label{fig:Linearize}
	\end{figure}
	\noindent
	Next up, we execute two commands in the command line of our workspace to import that model within Sisotool:
	\begin{itemize}
		\item \verb|fo = tf(linsys1)|
		\item \verb|sisotool(fo)|
	\end{itemize}
	This opens up Sisotool, with which we can create a controller for optimization purposes.
	\begin{figure}[H]
		\centering
		\includegraphics[width=1.0\linewidth]{sisotool4.png}
		\caption{Optimized step response shown in the Sisotool.}
		\label{fig:sisotool}
	\end{figure}
	\noindent
	Shown above in figure \ref{fig:sisotool} is an optimized step response in the Sisotool. To optimize it, we added an Integrator and Real Zero in the Bode Editor. This optimized model can be exported as a controller to the Matlab workspace as a variable \verb|C|.\newline
	\begin{figure}[H]
		\centering
		\includegraphics[width=1.0\linewidth]{closedLoop.png}
		\caption{Closed loop system with controller without voltage limits and quantizer.}
		\label{fig:closedLoop}
	\end{figure}
	\noindent
	Now we can modify our open-loop system by adding a \verb|LTI System| block to the input as shown above in figure \ref{fig:closedLoop}. Within this block we reference our variable \verb|C|. With that, the design of the speed controller is complete.
	\newpage
	
	\section*{Exercise 5c): Testing the designed controller and investigating the influence of voltage limits and quantizer-effects}
	To investigate the influence of voltage limits and quantizer-effects, we are going to add a \verb|Quantizer| and a \verb|Saturation| block to our closed loop model. This yields the following model as shown below in figure \ref{fig:closedLoopLimiters}:
	
	\begin{figure}[H]
		\centering
		\includegraphics[width=1.0\linewidth]{closedLoopLimiters.png}
		\caption{Closed loop system with controller with voltage limits and quantizer.}
		\label{fig:closedLoopLimiters}
	\end{figure}
	\noindent
	Hereby the output angular velocity as input of the controller is being quantized as steps of $5\frac{rad}{s}$ by a \verb|Quantizer| block. The output voltage of the controller gets limited by a \verb|Saturation| block to $\pm 12V$.\newline
	To compare both outputs of the closed-loop models with and without limiting blocks, we are going to overlay their step responses in a single plot, which is shown below in figure \ref{fig:controllerPlots}:
	
	\begin{figure}[H]
		\centering
		\includegraphics[width=1.0\linewidth]{controllerPlots.pdf}
		\caption{Step responses of both closed loop systems.}
		\label{fig:controllerPlots}
	\end{figure}
	\noindent
	The closed-loop system without limiters converges in just over 0.001 seconds to a steady value of $\omega$, whereas the system with limiters takes about 0.006 seconds. Also noticeable is the aspect that the graph of the system with limiters seems to contain some abrupt, step-like changes. Furthermore, both systems converge towards a slightly different value. This might be attributed to the \verb|Quantizer| block. \newline
	The Matlab code of the executing script is shown below.\newline\newline
	\noindent
	\textbf{Matlab Code Snippet:}
	\lstinputlisting[style=matlabstyle]{../exercise5_sudeesh/exercise5.m}
		
\end{document}
