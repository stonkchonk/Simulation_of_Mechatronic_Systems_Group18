\documentclass[10pt,a4paper]{article}
\usepackage[letterpaper,top=2cm,bottom=2cm,left=3cm,right=3cm,marginparwidth=1.75cm]{geometry}
\usepackage{amsmath}
\usepackage{graphicx}
\usepackage{url}
\usepackage{float}
\geometry{margin=1.8cm}

\usepackage{listings}
\usepackage{xcolor}

\lstdefinestyle{matlabstyle}{
	language=Matlab,
	basicstyle=\ttfamily\small,
	numbers=left,
	numberstyle=\tiny,
	backgroundcolor=\color{gray!5},
	frame=single,
	keywordstyle=\color{blue},
	commentstyle=\color{green!50!black},
	stringstyle=\color{orange},
	breaklines=true
}

\lstdefinestyle{cstyle}{
	language=C,
	basicstyle=\ttfamily\small,
	numbers=left,
	numberstyle=\tiny,
	backgroundcolor=\color{gray!5},
	frame=single,
	keywordstyle=\color{blue},
	commentstyle=\color{green!50!black},
	stringstyle=\color{orange},
	breaklines=true
}


\usepackage{hyperref}
\hypersetup{
	colorlinks=true,
	linkcolor=blue,
	filecolor=magenta,      
	urlcolor=cyan,
	pdftitle={Overleaf Example},
	pdfpagemode=FullScreen,
}


\title{Simulation of Mechatronic Systems \\\large Exercise 6 protocol - Building Simulink-Blocks in C}
\IfFileExists{authors.tex}{
	% authors.tex
\author{
	Alessandro Rotondi - 27445613 \\
	Frederic Heil - 28742953 \\
	Nithin Badri - 17546426 \\
	Sudeesh Suragari - 14646468
} % this pulls in authors.tex
}{
	\author{please create an authors.tex file}
}

\begin{document}
	\maketitle
	\section*{Exercise 6a): Implement the model as an S-Function using the S-function builder. Investigate the generated c source code.}
	In this exercise we aim to build our dc-motor model using a C based implementation approach. For this, Matlab provides us with the S-function builder. The corresponding, relevant source code is provided below:\newline\newline
	\noindent
	\textbf{C Code Snippet:}
	\lstinputlisting[style=cstyle]{../ex6_sudeesh/wrapper_functions.c}
	\noindent
	For our implementation, the beforehand generated functions \verb|exer6com_Outputs_wrapper| and \verb|exer6com_Derivatives_wrapper| are of interest. Within the first one, we assign the output values for the motor current $i$ and the angular speed $\omega$.\newpage
	
	\section*{Exercise 6b): Testing the block and comparing the results with the signal-flow model}
	Next up, we can test the C implementation within a designated Simulink block called \verb|S-Function Builder|. The Simulink model now looks like the following figure \ref{fig:sFunctionBlock} as shown below:\newline\newline
	\textbf{Simulink Model:}
	\begin{figure}[H]
		\centering
		\includegraphics[width=0.8\linewidth]{sFunctionBlock.png}
		\caption{S-Function block encompasses entire open loop model of a dc-motor.}
		\label{fig:sFunctionBlock}
	\end{figure}
	\noindent
	This large block corresponds to all the blue-marked blocks of the exercise description. To test it, we run the following Matlab script, which generates plots which can be compared to the previous results of exercise 4.\newline\newline
	\noindent
	\textbf{Matlab Code Snippet:}
	\lstinputlisting[style=matlabstyle]{../ex6_sudeesh/exer6code.m}
	This code snipped produces the following output plot of current $i$ and angular speed $\omega$ subject to time.
	\begin{figure}[H]
		\centering
		\includegraphics[width=0.8\linewidth]{sFunctionResponses.pdf}
		\caption{Step response of S-Function block.}
		\label{fig:sFunctionResponses}
	\end{figure}
	\noindent
	This result does in fact not seem to match very well with the results of exercise 4. Due to time constraints we have to leave it at that unfortunately.\newpage
	
	\section*{Exercise 6c): Creating our own symbol for the block}
	Following the Matlab documentation page \href{https://de.mathworks.com/help/simulink/ug/creating-subsystems.html}{Group Blocks into Subsystems}, we can define our own subsystem custom block. This block encompasses once again the functionalities of the dc-motor open loop model and looks like the following figure \ref{fig:subsystem} in Simulink:\newline\newline
	\textbf{Simulink Model:}
		
	\begin{figure}[H]
		\centering
		\includegraphics[width=0.8\linewidth]{subsystem.png}
		\caption{Subsystem block encompassing all the functionalities of the dc-motor open loop model.}
		\label{fig:subsystem}
	\end{figure}
	\noindent
	This subsystem can also be expanded to show all its inner workings.
	
	\section*{Exercise 6d): Investigating further examples of s-blocks inside the S-Function Demo library}
	Apparently, S-Functions, specifically their underlying code, can be written in multiple programming languages. Matlab allows for C, C++, Fortran and the Matlab language itself.
	\begin{figure}[H]
		\centering
		\includegraphics[width=0.8\linewidth]{examples.png}
		\caption{S-Function demo library.}
		\label{fig:examples}
	\end{figure}
	\noindent
	The shown sections contain subsections of multiple examples of complex systems compressed into a single block.
	
\end{document}

