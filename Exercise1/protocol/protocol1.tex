\documentclass[10pt,a4paper]{article}
\usepackage[letterpaper,top=2cm,bottom=2cm,left=3cm,right=3cm,marginparwidth=1.75cm]{geometry}
\usepackage{amsmath}
\usepackage{graphicx}
\usepackage{url}
\usepackage{float}
\geometry{margin=1.8cm}

\usepackage{listings}
\usepackage{xcolor}

\lstdefinestyle{matlabstyle}{
	language=Matlab,
	basicstyle=\ttfamily\small,
	numbers=left,
	numberstyle=\tiny,
	backgroundcolor=\color{gray!5},
	frame=single,
	keywordstyle=\color{blue},
	commentstyle=\color{green!50!black},
	stringstyle=\color{orange},
	breaklines=true
}


\title{Exercise 1 protocol}
\author{Member 1 - Matr.Nr. \\
Member 2 - Matr.Nr. \\
Member 3 - Matr.Nr. \\
Member 4 - Matr.Nr.}

\begin{document}
	\maketitle
	\section*{Exercise 1a)}
	
	\begin{figure}[H]
		\includegraphics[width=1.0\linewidth]{1a.pdf}
		\caption{Plots of trigonometric functions.}
		\label{fig:1a}
	\end{figure}
	
	As shown in Figure \ref{fig:1a} above, several trigonometric functions have been plotted into two different graphs, one for $sin$, $cos$ and $tan$ and one for their corresponding inverse functions. We chose a linspace $X$ ranging from $-\pi$ to $\pi$ with 100 steps.\newline
	Dedicated Matlab code below:
	
	\lstinputlisting[style=matlabstyle]{../exercise_1_a.m}
	
	\section*{Exercise 1b)}
	
	\begin{figure}[H]
		\includegraphics[width=1.0\linewidth]{1b.pdf}
		\caption{Calculated minimum distance between two lines in three dimensional space.}
		\label{fig:1b}
	\end{figure}
	
	Figure \ref{fig:1b} shows the minimum distance between the two lined defined by the four points $A$, $B$, $C$ and $D$.\newline
	At first we can calculate the directional vector for the two straight lines $AB$ and $CD$ as follows:
	\begin{equation}
		\begin{aligned}
			V_1 = B-A \\
			V_2 = D -C
		\end{aligned}
	\end{equation}
	
	With this we can define the two line equations $L_1(t)$ and $L_2(u)$ as follows:
	
	\begin{equation}
		\begin{aligned}
			L_1(t) = A +t*V_1 \\
			L_2(u) = C +u*V_2
		\end{aligned}
	\end{equation}
	
	Next up, we define the difference between support vectors $A$ and $C$ as $P_1P_{2diff}$ as:
	
	\begin{equation}
		P_1P_{2diff} = A - C
	\end{equation}
	
	\lstinputlisting[style=matlabstyle]{../exercise_1_b.m}
	
	\section*{Exercise 1c)}
	\lstinputlisting[style=matlabstyle]{../exercise_1_c.m}
	\section*{Exercise 1e)}
	As there is no d) exercise, we continue on to exercise e).
	\lstinputlisting[style=matlabstyle]{../exercise_1_e.m}
\end{document}
