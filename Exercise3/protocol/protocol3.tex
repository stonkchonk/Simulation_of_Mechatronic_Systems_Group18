\documentclass[10pt,a4paper]{article}
\usepackage[letterpaper,top=2cm,bottom=2cm,left=3cm,right=3cm,marginparwidth=1.75cm]{geometry}
\usepackage{amsmath}
\usepackage{graphicx}
\usepackage{url}
\usepackage{float}
\geometry{margin=1.8cm}

\usepackage{listings}
\usepackage{xcolor}

\lstdefinestyle{matlabstyle}{
	language=Matlab,
	basicstyle=\ttfamily\small,
	numbers=left,
	numberstyle=\tiny,
	backgroundcolor=\color{gray!5},
	frame=single,
	keywordstyle=\color{blue},
	commentstyle=\color{green!50!black},
	stringstyle=\color{orange},
	breaklines=true
}

\usepackage{hyperref}
\hypersetup{
	colorlinks=true,
	linkcolor=blue,
	filecolor=magenta,      
	urlcolor=cyan,
	pdftitle={Overleaf Example},
	pdfpagemode=FullScreen,
}


\title{Simulation of Mechatronic Systems \\\large Exercise 3 protocol}
\IfFileExists{authors.tex}{
	% authors.tex
\author{
	Alessandro Rotondi - 27445613 \\
	Frederic Heil - 28742953 \\
	Nithin Badri - 17546426 \\
	Sudeesh Suragari - 14646468
} % this pulls in authors.tex
}{
	\author{please create an authors.tex file}
}

\begin{document}
	\maketitle
	\section*{Exercise 3a): Simulating the step response of a transfer function}
	In the first sub task of this exercise, we want to simulate the step response of the transfer-function \textbf{PT1} which can be defined as follows:\newline
	\begin{equation}\label{eqPT1}
			G(s)=\frac{K}{1+T*s}
	\end{equation}
	\noindent
	Hereby, variables $K$ and $T$ are to be defined as $K=5$ and $T=4$. Furthermore, the differential equation of \textbf{PT1} is given as follows:\newline
	\begin{equation}\label{diffEqPT1}
		\dot{x}=\frac{1}{T}(K*u-x)
	\end{equation}
	\noindent
	Now that this has been established, we can build equation \ref{diffEqPT1} within Simulink which yields the following model:
	\begin{figure}[H]
		\centering
		\includegraphics[width=0.8\linewidth]{simulink_3a.png}
		\caption{Differential equation of \textbf{PT1} conveyed as Simulink model.}
		\label{fig:simulink3a}
	\end{figure}
	\noindent
	As shown in the figure above, the value of $u$ gets simulated by a step block. Both simulated values of $u$ and $x$ get forwarded to the scope block, which lets us examine their values within a Matlab script. Something that is not directly apparent from the graphic is the simulation run time, which has been set to 50 seconds.\newline
	Within our Matlab code, we can set and access the Simulink simulation as demonstrated in the code snippet below:\newline\newpage
	\noindent
	\textbf{Matlab Code Snippet:}
	\lstinputlisting[style=matlabstyle, linerange={1, 3-9, 37-39, 44-49}]{../a_exercise3.m}
	Hereby, we can access concrete scope variables using \texttt{plotG1\_De}, a logging value which has been defined within our Simulink model and is not directly apparent from any graphic.\newline
	The plot code of the step response yields the following graphic \ref{fig:plotStepResponse}:
	
	\begin{figure}[H]
		\centering
		\includegraphics[width=0.8\linewidth]{stepResponse3aDiffEq.pdf}
		\caption{Step response of our Simulink model plotted within Matlab.}
		\label{fig:plotStepResponse}
	\end{figure}
	\noindent
	As expected, the output \texttt{out} of the step function \textbf{PT1} converges at a value of 5, which corresponds to the value of parameter $K$.
	
	\section*{Exercise 3b): Building a one-mass system in Simulink}
	
	
	\section*{Exercise 3c): Building and optimizing a two-mass system in Simulink}
		
\end{document}
